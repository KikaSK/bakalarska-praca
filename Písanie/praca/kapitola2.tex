
\chapter{Numerické metódy}
\label{kap:numeric_methods}

V tejto kapitole si uvedieme numerické metódy zaoberajúce sa hľadaním riešenia rovnice $f(x) = 0$.
Presnejšie majme danú \textit{spojitú funkciu} $f: M \to \mathbb{R}, M \subset \mathbb{R}$. Hľadáme bod
$\bar{x} \in R$ taký, že $f(\bar{x}) = 0$.

Naše numerické metódy túto úlohu riešia len približne, avšak s ľubovoľnou presnosťou.

\section{Newtonova-Raphsonova metóda}

\textit{Newtonova-Raphsonova metóda}, známa tiež ako \textit{Newtonova metóda} je iteračná metóda používaná na riešenie
úlohy nášho typu. Snažíme sa skonštruovať postupnosť $\{x_n\}_{n=0}^\infty$ takú, že $x_n \to \bar{x}$.
Podľa Newtonovej-Raphsonovej metódy túto požiadavku spĺňa postupnosť bodov
$$x_0 \in \mathbb{R}, \,\,\,\,\,\,\,\,\, x_{n+1} = x_n - \frac{f(x_n)}{f'(x_n)}, \,\,\,\,\,\,\,\,\, n = 0, 1, 2, ...$$

Na obrázku \ref{obr:newton_raphson} môžeme vidieť príklad ako táto metóda funguje. Majme zadanú funkciu
$f(x)$, pre ktorú hľadáme jej koreň. Ide o priesečník jej grafu s osou $x$. Pre štartovací parameter
$x_0$ preložíme bodom $(x_0, f(x_0))$ dotyčnicu ku grafu. Ďalšiu iteráciu nájdeme ako priesečník tejto 
dotyčnice s osou $x$. Na obrázku môžeme vidieť štartovací bod $(x_0, 0)$ a prvé tri iterácie.

\begin{figure}
    \centerline{\includegraphics[width=0.8\textwidth]{images/newton_raphson}}
    \caption[Newtonova-Raphsonova metóda]{Newtonova-Raphsonova metóda na hľadanie koreňa funkcie $f(x)$.}
    %id obrazku, pomocou ktoreho sa budeme na obrazok odvolavat
    \label{obr:newton_raphson}
\end{figure}

Zastavovacích kritérií potrebujeme viacero. Ak je $f'(x_n) = 0$, je dotyčnica rovnobežná
s osou $x$, a teda neexistuje priesečník s osou $x$. Druhé kritérium je počet iterácií. 
Posledné zastavovacie kritérium je splnené v prípade, že $|f(x_n)|$ je dostatočne 
malé, a teda sme veľmi blízko koreňa. Posledné kritérium sa dá tiež preformulovať na $|x_{n+1} - x_n|$ 
je dostatočne malé.

Nevýhodou Newtonovej-Raphsonovej metódy je potreba poznať derivácie $f'$ v skúmaných bodoch, ale aj to, 
že negarantuje nájdenie
najbližšieho koreňa. Taktiež sa môže stať, že sa táto metóda zacyklí alebo oddiverguje. 
Príklad zacyklenia metódy môžeme vidieť
na obrázku \ref{obr:cyclic_newton_raphson}.

\begin{figure}
    \centerline{\includegraphics[width=0.8\textwidth]{images/cyclic_newton_raphson}}
    \caption[Zacyklenie Newtonovej-Raphsonovej metódy]{Zacyklenie Newtonovej-Raphsonovej metódy.}
    %id obrazku, pomocou ktoreho sa budeme na obrazok odvolavat
    \label{obr:cyclic_newton_raphson}
\end{figure}
\iffalse

\section{Metóda sečníc}

\textit{Metóda sečníc} je akási modifikácia \textit{Newtonovej-Raphsonovej metódy}. 
Môžeme ju použiť napríklad v prípade keď nepoznáme predpis $f'(x)$. 
V tejto metóde dotyčnicu v bode $x_n$ - $f'(x_n)$ nahradíme priamkou 
predchádzajúcou cez body $f(x_n), f(x_{n-1})$ - $\frac{f(x_n) - f(x_{n-1})}{x_n - x_{n-1}}$. 
Túto priamku nazývame \textit{sečnica}. 
Na rozdiel od Newtonovej-Raphsonovej metódy však potrebujeme až dva štartovacie body.

Iterácie v metóde sečníc získame predpisom
$$ x_{n+1} = x_n - \frac{f(x_n)}{\frac{f(x_n) - f(x_{n-1})}{x_n - x_{n-1}}}.$$

Na obrázku \ref{obr:secant_method} môžeme vidieť vizualizáciu tejto metódy.

\begin{figure}
    \centerline{\includegraphics[width=0.8\textwidth]{images/secant_method}}
    \caption[Metóda sečníc]{Metóda sečníc na hľadanie koreňa funkcie $f(x)$.}
    %id obrazku, pomocou ktoreho sa budeme na obrazok odvolavat
    \label{obr:secant_method}
\end{figure}

Nevýhodou tejto metódy je, že sa môže tak isto ako Newtonova-Raphsonova metóda zacykliť, teda 
opäť potrebujeme ako zastavovacie kritérium aj počet iterácii. Taktiež konverguje
pomalšie ako Newtonova-Raphsonova metóda.

\section{Zjednodušená Newtonova metóda - TODO asi vyhodiť preč}

\textit{Zjednodušená Newtonova metóda} sa môžem použiť v prípade, ak \textit{Newtonova-Raphsonova metóda} zlyhá na prvom 
zastavovacom kritériu, teda na nulovosti derivácie. Táto metóda nahrádza $f'(x_n)$ dotyčnicou v začiatočnom bode $f'(x_0)$.  
\fi

\section{Metóda bisekcie}

\textit{Metóda bisekcie}, známa tiež ako \textit{Metóda polenia intervalu} je založená 
na pozorovaní, že v okolí jednoduchého koreňa funkčné hodnoty spojitej funkcie menia znamienko. 
Je to jedna z najpomalších, ale zároveň najbezpečnejších metód.

Majme spojitú funkciu $f(x)$ a body $a$ a $b$, také, že $f(a) \cdot f(b) < 0$, teda 
funkčné hodnoty majú v týchto bodoch opačné znamienka. Metóda bisekcie delí tento 
interval na polovicu a na základe znamienka funkčnej hodnoty v novom bode $s = (a+b)/2$
rozhodne, či sa koreň nachádza na intervale $(a, s)$ alebo na intervale $(s, b)$. 
Ak platí $f(s) = 0$, potom má funkcia v danom bode koreň. Daný postup sa opakuje
pokiaľ nie je splnené niektoré zo zastavovacích kritérií. Na obrázku \ref{obr:bisection}
môžeme vidieť štartovacie body $x_0$ a $x_1$ a tiež prvé 3 iterácie. Vodorovnými
úsečkami so šípkami je znázornená postupná voľba intervalov. 

\begin{figure}
    \centerline{\includegraphics[width=0.8\textwidth]{images/bisection}}
    \caption[Metóda bisekcie]{Metóda bisekcie na hľadanie koreňa funkcie $f(x)$.}
    %id obrazku, pomocou ktoreho sa budeme na obrazok odvolavat
    \label{obr:bisection}
\end{figure}

Ak sa na intervale $(a, b)$ nachádza viac koreňov, metóda bisekcie nájde aspoň jeden z nich.

Zjavnou nevýhodou tejto metódy je podmienka štartovacích bodov a zároveň rýchlosť algoritmu. 
Veľkou výhodou však je, že ak vieme, že na intervale $(a, b)$ sa nachádza práve jeden koreň, 
touto metódou ho nájdeme. \textit{Metóda bisekcie} sa takisto nemôže zacykliť.

