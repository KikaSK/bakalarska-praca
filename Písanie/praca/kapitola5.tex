\chapter{Implementácia}
\label{kap:implementation}

V tejto kapitole popíšeme presnejšie implementáciu algoritmu opísaného v kapitole ref{kap:algoritmus}.
Algoritmus sme naimplementovali v jazyku C++ s využitím knižnice GiNaC. Všetky desatinné čísla
reprezentujeme ako typ \textit{numeric} z knižnice GiNaC. Výhodou tohto typu je, že ho vieme 
reprezentovať s ľubovoľnou presnosťou. Z knižnice taktiež využívame typ \textit{ex}, využívaný na 
reprezentáciu výrazov a jeho metódu na počítanie parciálnej derivácie.

\section{Dizajn}

V algoritme využívame viaceré štruktúry, naprogramované ako \textit{triedy}. V tejto podkapitole
vymenujeme tieto triedy a niektoré dôležité metódy.

\begin{itemize}
    \item{
        \textit{Point}

        Trieda používaná na reprezentáciu trojrozmerného bodu.
    }
    \item{
        \textit{Vector}
        
        Trieda používaná na reprezentáciu trojrozmerného vektora. 
    }
    \item{
        \textit{Edge}

        Trieda používaná na reprezentáciu hrany danej dvomi bodmi.
    }
    \item{
        \textit{Triangle}

        Trieda používaná na reprezentáciu trojuholníka daného tromi bodmi.
    }
    \item{
        \textit{Function}

        Trieda používaná na reprezentáciu vstupnej funkcie zadanej implicitne. 
    }
    \item{
        \textit{BoundingBox}

        Trieda používaná na ukladanie informácii o ohraničujúcej obálke pre ohraničenú trianguláciu.
    }
    \item{
        \textit{Mesh}

        Trieda používaná na ukladanie dát o meshi. Táto trieda má implementovanú napríklad metódu
        $check\_Delaunay$, ktorá overuje \textit{Delaunayovu podmienku} spolu s podmienkou
        na vzdialenosť ťažiska. Takisto metódu $get\_breakers$, ktorá nájde hraničné body 
        vnútri \textit{Delaunayovej gule} pre zadaný trojuholník $T$.
    }
    \item{
        \textit{BasicAlgorithm}

        Najdôležitejšia trieda, v ktorej sa odohráva väčšina výpočtov. Najdôležitejšie metódy 
        tejto triedy sú
        \begin{itemize}
            \item{
                $first\_part$
                
                Metóda, v ktorej sa odohráva základná štruktúra prvej časti algoritmu opísaného v kapitole 
                \ref{kap:first_part_of_algorithm}. Vyberá hranu $E=(x_i, x_j)$ zo zoznamu hraničných 
                hrán a následne pre túto hranu volá metódu \textit{step}.
            }
            \item{
                $step$

                Metóda, v ktorej sa v správnom poradí volajú ďalšie metódy tak, ako boli opísané v 
                kapitole \ref{kap:finding_new_vertex}.
                Ak sa nám v žiadnej z týchto metód neporadí nájsť nový trojuholník na konci kroku 
                označíme hranu $E = (x_i, x_j)$ za skontrolovanú.
            }
            \item{
                $get\_projected$

                Metóda, ktorá vypočíta bod $x_{new}$ pre hraničnú hranu $E$.
            }
            \item{
                $find\_prev\_next$

                Metóda, ktorá k hrane $E$ nájde jej susedov. 
            }
            \item{
                $fix\_same\_points$

                Metóda ekvivalentná s krokom číslo $3$ v kapitole \ref{kap:finding_new_vertex}, 
                spájajúca body premietnuté veľmi blízko už existujúceho vrchola s týmto vrcholom.
            }
            \item{
                $basic\_triangle$

                Metóda ekvivalentná s krokom číslo $4$ v kapitole \ref{kap:finding_new_vertex},
                vytvára trojuholník ak je pri hrane diera tvaru trojuholníka.
            }
            \item{
                $fix\_breakers$

                Metóda ekvivalentná s krokom číslo $5$ v kapitole \ref{kap:finding_new_vertex},
                pokúšajúca sa vytvoriť trojuholníky s bodmi nachádzajúcimi sa v 
                \textit{Delaunayovej guli}.
            }
            \item{
                $fix\_proj$

                Metóda ekvivalentná s krokmi číslo $6-8$ v kapitole \ref{kap:finding_new_vertex}.
                V tejto metóde sa pokúšame spájať bod s vrcholmi bližšími ako $0.4 \, a$. Následne 
                s krajnými bodmi najbližšej hrany, tak ako v kroku $7$ a nakoniec vytvárame 
                trojuholník $T_{new}$. Pri ohraničenej triangulácii využívame
                metódu na prichytávanie vrcholov
                k obálke, keďže pri tvorbe nového vrchola $x_{new}$ môžeme vyjsť von z 
                ohraničujúcej obálky.
            }
            \item{
                $fix\_prev\_next$

                Metóda ekvivalentná s krokom číslo $9$ v kapitole \ref{kap:finding_new_vertex},
                pokúšajúca sa vytvoriť trojuholníky so susednými vrcholmi $x_{prev}$ a $x_{next}$.
            }
            \item{
                $second_part$

                Metóda, v ktorej sa odohráva základná štruktúra druhej časti algoritmu uzatvárajúceho
                diery \ref{kap:second_part_of_algorithm}. Vyberá hranu $E=(x_i, x_j)$ zo zoznamu hraničných 
                hrán a následne pre túto hranu volá metódu $fix\_holes$.
            }
            \item{
                $fix\_holes$

                Metóda, v ktorej sa v správnom poradí volajú ďalšie metódy tak ako boli opísané v 
                kapitole \ref{kap:second_part_of_algorithm}. Tieto metódy sú rovnaké ako pre prvú
                časť algoritmu avšak vypúšťame v nich overovanie \textit{Delaunayovej podmienky} a
                podmienky s blízkosťou ťažiska.
                Ak sa nám v žiadnej z týchto metód neporadí nájsť nový trojuholník na konci kroku 
                označíme hranu $E = (x_i, x_j)$ za skontrolovanú.
            }
            \item{
                $fix\_corners$

                V tejto metóde riešime problém \textit{odseknutých} rohov, ktorý sme spomínali v
                kapitole \ref{kap:bounded_triangulation}.
                Metódu voláme po skončení základnej časti algoritmu aj časti uzatvárajúcej diery.
            }
        \end{itemize}
        }

\end{itemize}
Okrem tried a ich metód sme niektoré algoritmy naimplementovali ako funkcie. 
        Niektoré dôležité funkcie:
        \begin{itemize}
            \item{
                $Newton\_Raphson$

                Newton-Rapsonova metóda prezentovaná v kapitole \ref{kap:numeric_methods}.
            }
            \item{
                $Bisect$

                Metóda bisekcie prezentovaná v kapitole \ref{kap:numeric_methods}.
            }
            \item{
                $Bisection$

                Funkcia, v ktorej nájdeme druhý bod na opačnej strane povrchu a zavoláme
                funkciu $Bisect$.
            }
            \item{
                $project$

                Funkcia, ktorá z funkcie zadanej implicitne a smerového vektora priemetu
                vypočíta vstupnú funkciu pre numerické metódy. Volá funkciu $Newton\_Raphson$
                a v prípade neúspechu funkciu $Bisection$, vracia premietnutý bod.
            }
            \item{
                $line\_point\_dist$

                Počíta vzdialenosť hrany a bodu tak ako sme definovali v definícii 
                \ref{def:segment_point_distance}.
            }
            \item{
                $angle$

                Počíta uhol dvoch vektorov vzhľadom na trojuholník ako sme definovali v kapitole 
                \ref{kap:triangle_conditions} v bode $2$.
            }
        \end{itemize}
\section{Štruktúra}

Pokým nie sú všetky hrany skontrolované, opakujeme kroky $1-3$.
\begin{enumerate}
    \item{
        Vyberieme zo zoznamu neskontrolovaných hrán hranu E.
    }
    \item{
        \begin{itemize}
            \item{
                if $(fix\_same\_points)$ \, $return$
            }
            \item{
                if $(basic\_triangle)$ \, $return$
            }
            \item{
                if $(fix\_breakers)$ \, $return$
            }
            \item{
                if $(fix\_proj)$ \, $return$
            }
            \item{
                if $(fix\_prev\_next)$ \, $return$
            }
        \end{itemize}
    }
    \item{
        Vložíme hranu $E$ do zoznamu skontrolovaných hrán.
    }
    \item{
        Vložíme všetky skontrolované hrany do zoznamu neskontrolovaných hrán a 
        zoznam skontrolovaných hrán vymažeme. Pokým nie je zoznam neskontrolovaných hrán prázdny
        opakujeme kroky $5-7$, pričom nekontrolujeme \textit{Delaunayovu podmienku}.
    }
    \item{
        Vyberieme zo zoznamu neskontrolovaných hrán hranu E.
    }
    \item{
        \begin{itemize}
            \item{
                if $(basic\_triangle)$ \, $return$
            }
            \item{
                Nájdeme najbližší bod, ak spĺňa podmienky, pridáme ho.
            }
            \item{
                if $(fix\_breakers)$ \, $return$
            }
            \item{
                if $(fix\_prev\_next)$ \, $return$
            }
        \end{itemize}
    }
    \item{
        Vložíme hranu $E$ do zoznamu skontrolovaných hrán.
    }
    \item{
        $fix\_corners$
    }
\end{enumerate}

\section{Bližšie objasnenie niektorých dôležitých metód}
\label{kap:important_methods}
\begin{itemize}
    \item{
        metóda $outside\_normal$ v triede $Function$
        
        Táto metóda vypočíta normálu trojuholníka ukazujúcu \textit{von} z plochy, 
        pričom \textit{von} pre nás znamená do priestoru, kde $F(x)>0$.
        Je založená na predpoklade, že funkcia je hladká. Pre trojuholník
        $T = (x_i, x_j, x_k)$ a jednu z jeho normál $\overrightarrow{n}$
        zadefinujeme vektory $\overrightarrow{n}_{\varepsilon}^+ = \varepsilon \, \overrightarrow{n}$
        a $\overrightarrow{n_{\varepsilon}^-} = - \varepsilon \, \overrightarrow{n}$.
        Tieto vektory umiestnime do jedného z vrcholov trojuholníka $T$, označme tento vrchol $x$.
        Keďže funkcia je hladká, tak pre dostatočne malé $\varepsilon$ bude jeden z
        bodov $x + \overrightarrow{n}_{\varepsilon}^+$ a $x + \overrightarrow{n}_{\varepsilon}^-$ 
        pod povrchom plochy a jeden nad povrchom plochy. 
        $\varepsilon$ volíme ako $0.1 \, a$. Táto dĺžka by mala byť dostatočne malá vďaka predpokladu, 
        že $a$ je dostatočne 
        malé na to aby sme mohli trojuholníkom s dĺžkou strany $a$ triangulovať plochu.
        Vizualizáciu $2D$ ekvivalentu tejto metódy môžeme vidieť na obrázku \ref{obr:outside_normal}.
        Červenou sú vyznačené body pod povrchom a zelenou body nad povrchom. Vďaka tomu sa vieme 
        rozhodnúť, ktorá normála ukazuje von z povrchu, na obrázku označená zelenou farbou.

        \begin{figure}
            \centerline{\includegraphics[width=0.55\textwidth]{images/outside_normal}}
            \caption[Počítanie normály trojuholníka ukazujúcej von z plochy]
            {Počítanie normály trojuholníka ukazujúcej von z plochy.}
            %id obrazku, pomocou ktoreho sa budeme na obrazok odvolavat
            \label{obr:outside_normal}
        \end{figure}
    }
    \item{

        metóda $crop\_to\_box$ v triede $BoundingBox$
        
        V tejto metóde riešime prichytávanie bodov na ohraničujúcu obálku.
        Spomínali sme, že sme uvažovali o dvoch spôsoboch.
        Prvý spôsob, nad ktorým sme uvažovali bolo premietanie blízkych bodov vnútri obálky aj 
        bodov vonku z obálky na najbližší bod na obálke. Teda premietanie v smere normál stien obálky.
        Tento prístup sme otestovali a ukázalo sa, že má vážny nedostatok a to ten, že premietnuté 
        vrcholy boli často príliš ďaleko od plochy a vytvárali nepresný a zúbkovaný okraj plochy. 
        Z tohto dôvodu
        sme zvolili iný prístup, pri ktorom takisto nie je garantovaná poloha bodu na ploche, avšak 
        chyba je oveľa menšia. Tento prístup je premietanie bodu $x_{new}$ na steny obálky v smere 
        ťažnice trojuholníka $T = (x_i, x_j, x_{new})$ vedúcej cez bod $x_{new}$.

        Na obrázku \ref{obr:crop_to_box} môžeme vidieť oba prístupy. Premietanie v smere normály 
        na obrázku $a)$ a $b)$, premietanie v smere ťažnice na obrázku $c)$ a $d)$. Na prvý pohľad 
        výsledky nevyzerajú veľmi rozdielne, avšak v kapitole \ref{kap:results} uvidíme pozitívny vplyv na výslednú 
        trianguláciu. 
        
        \begin{figure}
            \centerline{\includegraphics[width=1\textwidth]{images/crop_to_box}}
            \caption[Orezávanie na ohraničujúcu obálku]
            {Prichytávanie na najbližší bod(vľavo), premietanie na obálku v smere ťažnice(vpravo)}
            %id obrazku, pomocou ktoreho sa budeme na obrazok odvolavat
            \label{obr:crop_to_box}
        \end{figure}
    }
    \item{
        metóda $find\_prev\_next$ v triede $BasicAlgorithm$

        Suseda $x_{prev}$ definujeme ako 
                hraničný vrchol, susedný s vrcholom $x_i$, taký, že uhol $\alpha$ dvoch vektorov
                $\overrightarrow{x_i x_j}$ a $\overrightarrow{x_i x_{prev}}$ vzhľadom 
                na susedný trojuholník $N$ ku hrane $E$ je najmenší. Uhol $\alpha$ definujeme podobne
                ako uhol $\beta$ v kapitole \ref{kap:triangle_conditions} v bode 2 s malou zmenou. 
                \[ 
                \alpha = \left\{
                \begin{array}{ll}
                    \beta + 2 \pi & \beta \in \langle -\pi, 0 \rangle\\
                    \beta & \beta \in \langle 0, \pi \rangle\\
                \end{array} 
                \right. 
                \]
                
                Suseda $x_{next}$ definujeme analogicky pre vrchol $x_j$ a uhol vektorov 
                $\overrightarrow{x_j x_i}$ a $\overrightarrow{x_j x_{next}}$.

                Ilustráciu môžeme vidieť na obrázku \ref{obr:find_prev_next}, vrchol $x_j$
                má len jedného suseda medzi hraničnými vrcholmi, teda $x_{next} = x_n$. 
                Avšak vrchol $x_i$ má troch
                susedov $x_{p_1}, x_{p_2}, x_{p_3}$. Najmenší uhol $\alpha$ je pri vrchole $x_{p_1}$,
                preto tento vrchol označíme za $x_{prev}$.

                \begin{figure}
                    \centerline{\includegraphics[width=0.5\textwidth]{images/find_prev_next}}
                    \caption[Hľadanie susedných hraničných vrcholov]
                    {Hľadanie susedných hraničných vrcholov.}
                    %id obrazku, pomocou ktoreho sa budeme na obrazok odvolavat
                    \label{obr:find_prev_next}
                \end{figure}
    }
    \item{
        metóda $fix\_corners$ z triedy $BasicAlgorithm$

        Nazvime ohraničujúce hrany také, ktorých oba konce ležia na ohraničujúcej obálke.
                
                Na začiatku metódy nájdeme medzi ohraničujúcimi hranami tie, ktorých konce neležia
                na tej istej stene. Na to využívame nasledujúci postup. Pre oba vrcholy ohraničujúcej 
                hrany vypočítame číslo $f$ reprezentujúce množinu stien, na ktorých daný vrchol leží. 
                Ak si ho predstavíme ako binárne číslo, má jednotku na 1. bite, ak vrchol leží na stene 
                $x_{min}$, na 2. bite, ak leží na stene $x_{max}$, 3. bit patrí stene $y_{min}$, 4. bit
                stene $y_{max}$, 5. bit stene $z_{min}$ a napokon 6. bit stene $z_{max}$. Na ostatných 
                bitoch zostáva 0. 
                
                Potom ak je bitový $and$ týchto čísel pre oba vrcholy nenulový, znamená
                to, že majú nejakú spoločnú stenu. Ak je naopak nulový, neležia na spoločnej stene.
                Zaujímajú nás teda hrany, ktorých bitový $and$ čísel $f_1$ a $f_2$ pre ich krajné 
                vrcholy je 0. Príklad takéhoto výpočtu môžeme vidieť na obrázku \ref{obr:common_faces}, 
                v tomto prípade binárne čísla konvertujeme na decimálne \textit{odzadu}.

                V rámiku naľavo je bitový $and$ nenulový a keďže nenulová číslica je na 5. bite, body ležia
                na spoločnej stene $z_{min}$. V rámiku napravo je bitový $and$ nulový, teda body neležia
                na spoločnej stene.

                \begin{figure}
                    \centerline{\includegraphics[width=1\textwidth]{images/common_faces}}
                    \caption[Hľadanie spoločných stien ohraničujúcich hrán]
                    {Hľadanie spoločných stien ohraničujúcich hrán.}
                    %id obrazku, pomocou ktoreho sa budeme na obrazok odvolavat
                    \label{obr:common_faces}
                \end{figure}

                Pre tieto hrany chceme vytvoriť vrchol, ktorý leží na ploche a zároveň na hrane obálky.
                To dosiahneme premietnutím bodu ležiaceho na blízkej hrane na zadanú plochu v smere 
                smerového vektora tejto hrany. Ako tretiu zo súradníc bodu pre premietanie určíme 
                prislúchajúcu súradnicu stredu strany $E$. V prípade, že niektorý z vrcholov 
                $x_i$, $x_j$ leží na hrane obálky, vytvoríme nový bod vo vrchole tejto obálky. 
                Opísaný postup môžeme vidieť na obrázku \ref{obr:fix_corners}. Červenou farbou 
                je vyznačená hrana ležiaca na dvoch rôznych stenách, krížikom nový bod a zelenou farbou
                nový trojuholník.
                
                
                \begin{figure}
                    \centerline{\includegraphics[width=0.7\textwidth]{images/fix_corners}}
                    \caption[Prichytávanie na hrany a rohy ohraničujúcej obálky]
                    {Prichytávanie na hrany a rohy ohraničujúcej obálky.}
                    %id obrazku, pomocou ktoreho sa budeme na obrazok odvolavat
                    \label{obr:fix_corners}
                \end{figure}

    }
\end{itemize}