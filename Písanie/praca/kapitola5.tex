\chapter{Implementácia}
\label{kap:implemetation}

V tejto kapitole popíšeme presnejšie implementáciu algoritmu opísaného v kapitole TODO.
Algoritmus sme naimplementovali v jazyku C++ s využitím knižnice GiNaC. Všetky desatinné čísla
reprezentujeme ako typ \textit{numeric} z knižnice GiNaC. Výhodou tohto typu je, že ho vieme 
používať na ľubovoľnú presnosť. Z knižnice taktiež využívame typ \textit{exp}, využívaný na 
reprezentáciu výrazov.

\section{Dizajn kódu}

V algoritme využívame viaceré štruktúry, naprogramované ako \textit{triedy}. V tejto podkapitole
vymenujeme tieto triedy a niektoré ich dôležité metódy.

\begin{itemize}
    \item{
        \textit{Point}

        Trieda používaná na reprezentáciu trojrozmerného bodu.
    }
    \item{
        \textit{Vector}
        
        Trieda používaná na reprezentáciu trojrozmerného vektora. 
    }
    \item{
        \texit{Edge}

        Trieda používaná na reprezentáciu hrany danej dvomi bodmi.
    }
    \item{
        \textit{Triangle}

        Trieda používaná na reprezentáciu trojuholníka daného tromi bodmi.
    }
    \item{
        \textit{Function}

        Trieda používaná na reprezentáciu vstupnej funkcie danej implicitne.
        V tejto triede je zaujímavá metóda \textit{outside_normal}, ktorá
        vypočíta normálu trojuholníka ukazujúcu von z plochy. Táto metóda
        je založená na predpoklade, že funkcia je hladká.
    }
\end{itemize}