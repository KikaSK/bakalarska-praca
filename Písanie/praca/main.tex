\documentclass[12pt, twoside]{book}
%\documentclass[12pt, oneside]{book}  % jednostranna tlac
\usepackage[a4paper,top=2.5cm,bottom=2.5cm,left=3.5cm,right=2cm]{geometry}
\usepackage[utf8]{inputenc}
\usepackage[T1]{fontenc}
\usepackage{graphicx}
\usepackage{url}
\usepackage[hidelinks,breaklinks]{hyperref}
\usepackage[slovak]{babel} % vypnite pre prace v anglictine
\linespread{1.25} % hodnota 1.25 by mala zodpovedat 1.5 riadkovaniu

\usepackage{amssymb}
\usepackage[shortlabels]{enumitem}
\usepackage[utf8]{inputenc}
\usepackage{xstring}
\usepackage{mathtools}

\usepackage{tikz}
\usepackage{collcell}
\usepackage{relsize}

%moje priklazy
\newtheorem{definition}{Definícia}
\newtheorem{theorem}{Veta}
\newtheorem{rovnica}{}
\newtheorem{note}{Poznámka}

% -------------------
% --- Definicia zakladnych pojmov
% --- Vyplnte podla vasho zadania
% -------------------
\def\mfrok{2021}
\def\mfnazov{Algoritmy triangulácie implicitne definovanej plochy}
\def\mftyp{Bakalárska práca}
\def\mfautor{Kristína Korecová}
\def\mfskolitel{doc. RNDr. Pavel Chalmovianský, PhD. }

%ak mate konzultanta, odkomentujte aj jeho meno na titulnom liste
\def\mfkonzultant{tit. Meno Priezvisko, tit. }  

\def\mfmiesto{Bratislava, \mfrok}

% bioinformatici odkomentujú riadok s dvoma odbormi a iný program
\def\mfodbor{ Matematika }
%\def\mfodbor{ Informatika a Biológia } 
\def\program{ Matematika }
%\def\program{ Bioinformatika }

% Ak je školiteľ z FMFI, uvádzate katedru školiteľa, zrejme by mala byť aj na zadaní z AIS2
% Ak máte externého školiteľa, uvádzajte Katedru informatiky 
\def\mfpracovisko{ Katedra algebry a geometrie }

\begin{document}     
\frontmatter


% -------------------
% --- Obalka ------
% -------------------
\thispagestyle{empty}

\begin{center}
\sc\large
Univerzita Komenského v Bratislave\\
Fakulta matematiky, fyziky a informatiky

\vfill

{\LARGE\mfnazov}\\
\mftyp
\end{center}

\vfill

{\sc\large 
\noindent \mfrok\\
\mfautor
}

\cleardoublepage
% --- koniec obalky ----

% -------------------
% --- Titulný list
% -------------------

\thispagestyle{empty}
\noindent

\begin{center}
\sc  
\large
Univerzita Komenského v Bratislave\\
Fakulta matematiky, fyziky a informatiky

\vfill

{\LARGE\mfnazov}\\
\mftyp
\end{center}

\vfill

\noindent
\begin{tabular}{ll}
Študijný program: & \program \\
Študijný odbor: & \mfodbor \\
Školiace pracovisko: & \mfpracovisko \\
Školiteľ: & \mfskolitel \\
% Konzultant: & \mfkonzultant \\
\end{tabular}

\vfill


\noindent \mfmiesto\\
\mfautor

\cleardoublepage
% --- Koniec titulnej strany


% -------------------
% --- Zadanie z AIS
% -------------------
% v tlačenej verzii s podpismi zainteresovaných osôb.
% v elektronickej verzii sa zverejňuje zadanie bez podpisov
% v pracach v naglictine anglicke aj slovenske zadanie

\newpage 
\thispagestyle{empty}
\hspace{-2cm}\includegraphics[width=1.1\textwidth]{images/zadanie}

% --- Koniec zadania

\frontmatter

% -------------------
%   Poďakovanie - nepovinné
% -------------------
\setcounter{page}{3}
\newpage 
~

\vfill
{\bf Poďakovanie:} Na tomto mieste by som chcela poďakovať najmä môjmu školiteľovi,
doc. RNDr. Pavlovi Chalmovianskému, PhD., za cenné rady, aj za trpezlivosť s mojimi 
otázkami. Tiež by som chcela poďakovať mojej mame za podporu a starostlivosť. 
Ďakujem aj môjmu partnerovi Andrejovi 
Kormanovi za najlepšiu technickú podporu a trpezlivé (a občas aj menej trpezlivé) 
počúvanie mojich nápadov. Nakoniec by som chcela poďakovať mojej kamarátke a spolužiačke
Beate za to, že mi bola ten najlepší \textit{debuggovací obláčik}.

% --- Koniec poďakovania

% -------------------
%   Abstrakt - Slovensky
% -------------------
\newpage 
\section*{Abstrakt}

Triangulácia implicitne definovanej plochy je častý spôsob diskrétnej aproximácie
plochy. V tejto práci predstavujeme algoritmus na trianguláciu implicitne definovanej plochy
v regulárnych častiach plochy. Algoritmus trianguluje
konečné plochy, ale aj nekonečné plochy v ohraničenej časti euklidovského priestoru. 
Zameriavame sa na tvorbu kvalitnej triangulácie, ktorá má čo najpravidelnejšie
trojuholníky, využívajúc metódu sledovania plochy, pričom nekladieme dôraz na rýchlosť výpočtu. 
Implementujeme a porovnávame rôzne spôsoby orezávania na ohraničenú časť euklidovského priestoru. 
Vytvárame aj verzie algoritmu pre rôzne druhy adaptívnej triangulácie na trianguláciu plôch s rôznym 
zakrivením povrchu.

\paragraph*{Kľúčové slová:} implicitne definovaná plocha, triangulácia, algoritmus
% --- Koniec Abstrakt - Slovensky


% -------------------
% --- Abstrakt - Anglicky 
% -------------------
\newpage 
\section*{Abstract}

Triangulation of implicit surface is a common way of discrete approximation of the surface.
In this thesis we present an algorithm for triangulation of implicit surface in it's regular 
parts. The algorithm triangulates finite surfaces as well as infinite surfaces in bounded part
of Euclidean space. We focus on creating quality triangulation with as regular triangles as 
possible. We are using surface tracking method while not emphasizing the speed of calculation.
We implement and compare different methods for trimming the surface to bounded part of Euclidean 
space. We also create multiple adaptive versions of the algorithm for triangulation of surfaces 
with different surface curvature.  


\paragraph*{Keywords:} kľúčové slová po anglicky...

% --- Koniec Abstrakt - Anglicky

% -------------------
% --- Predhovor - v informatike sa zvacsa nepouziva
% -------------------
%\newpage 
%\thispagestyle{empty}
%
%\huge{Predhovor}
%\normalsize
%\newline
%Predhovor je všeobecná informácia o práci, obsahuje hlavnú charakteristiku práce 
%a okolnosti jej vzniku. Autor zdôvodní výber témy, stručne informuje o cieľoch 
%a význame práce, spomenie domáci a zahraničný kontext, komu je práca určená, 
%použité metódy, stav poznania; autor stručne charakterizuje svoj prístup a svoje 
%hľadisko. 
%
% --- Koniec Predhovor


% -------------------
% --- Obsah
% -------------------

\newpage 

\tableofcontents

% ---  Koniec Obsahu

% -------------------
% --- Zoznamy tabuliek, obrázkov - nepovinne
% -------------------

\newpage 

\listoffigures
\listoftables

% ---  Koniec Zoznamov

\mainmatter

%názvy kapitol

\input uvod.tex 

\input kapitola1.tex

\input kapitola2.tex

\input kapitola3.tex

\input kapitola4.tex

%\input kapitola5.tex

\input kapitola6.tex

\input zaver.tex

% -------------------
% --- Bibliografia
% -------------------


\newpage	

\backmatter

\thispagestyle{empty}
\clearpage

\bibliographystyle{plain}
\bibliography{literatura} 

%Prípadne môžete napísať literatúru priamo tu
%\begin{thebibliography}{5}
 
%\bibitem{br1} MOLINA H. G. - ULLMAN J. D. - WIDOM J., 2002, Database Systems, Upper Saddle River : Prentice-Hall, 2002, 1119 s., Pearson International edition, 0-13-098043-9

%\bibitem{br2} MOLINA H. G. - ULLMAN J. D. - WIDOM J., 2000 , Databasse System implementation, New Jersey : Prentice-Hall, 2000, 653s., ???

%\bibitem{br3} ULLMAN J. D. - WIDOM J., 1997, A First Course in Database Systems, New Jersey : Prentice-Hall, 1997, 470s., 

%\bibitem{br4} PREFUSE, 2007, The Prefuse visualization toolkit,  [online] Dostupné na internete: <http://prefuse.org/>

%\bibitem{br5} PREFUSE Forum, Sourceforge - Prefuse Forum,  [online] Dostupné na internete: <http://sourceforge.net/projects/prefuse/>

%\end{thebibliography}

%---koniec Referencii

% -------------------
%--- Prilohy---
% -------------------

%Nepovinná časť prílohy obsahuje materiály, ktoré neboli zaradené priamo  do textu. Každá príloha sa začína na novej strane.
%Zoznam príloh je súčasťou obsahu.
%
\input appendixA.tex

\input appendixB.tex

\end{document}






