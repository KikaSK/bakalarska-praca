\chapter*{Záver}  % chapter* je necislovana kapitola
\addcontentsline{toc}{chapter}{Záver} % rucne pridanie do obsahu
\markboth{Záver}{Záver} % vyriesenie hlaviciek

Cieľom práce bolo vybudovať algoritmus na spoľahlivú trianguláciu regulárnej implicitne definovanej plochy, 
s nízkou chybovosťou a vlastnosťami kvalitnej triangulácie. Algoritmusmal byť schopný triangulovať
uzavreté, konečné plochy, avšak aj nekonečné plochy s použitím ohraničujúcej obálky. Sekundárnym cieľom bolo
vyskúšať niektoré techniky adaptívnej triangulácie pre plochy obsahujúce oblasti s nižším aj vyšším zakrivením.

V kapitole \ref{kap:algoritmus} sme navrhli algoritmus spĺňajúci naše požiadavky. Tento algoritmus sme
naimplementovali v jazyku C++, a následne sme v kapitole \ref{kap:results} odprezentovali dosiahnuté výsledky
pri triangulácii konečných, uzavretých ale aj nekonečných, ohraničených plôch. Vyskúšali sme a porovnali dva 
rôzne prístupy orezávania plochy na ohraničujúcu obálku. Taktiež sme porovnali rôzne prístupy adaptívnej 
triangulácie. 

V práci sme sa nevenovali optimalizácii algoritmu a preto je pomerne časovo náročný. V budúcnosti by 
sa dal náš algoritmus časovo optimalizovať najmä použitím hešovacích tabuliek, v ktorých vieme vyhľadávať 
v amortizovanej časovej zložitosti $\mathcal{O}(1)$, pričom v našej implementácií vyhľadávame v časovej
zložitosti $\mathcal{O}(n)$. Vyhľadávanie tvorí veľkú časť výpočtového času algoritmu. Ďalej navrhujeme 
premyslenie dátovej štruktúry, založenej na rekurzívnom prerozdelení priestoru na 8 častí (octree), pre 
úspornejšie overovanie \textit{Delaunayovej podmienky} aj novej podmienky overujúcej blízkosť ťažiska.
Pri tvorbe nového trojuholníka totiž stačí overovať tieto podmienky pre blízke trojuholníky.

Budúca práca vedie k triangulácii plôch v okolí ich singulárnych bodov a kriviek, takisto intenzívnejšie 
preskúmanie algoritmov pre adaptívnu trianguláciu. Následný návrh algoritmu, by nemal byť limitovaný nevýhodou pretínania
trojuholníkov pri stretnutí trojuholníkov s veľmi rozdielnymi veľkosťami hrán. Taktiež by sme chceli navrhnúť spôsob 
prichytávania vrcholov na ohraničujúcu obálku kombinujúcu výhody oboch prezentovaných prístupov. Táto metóda by mala 
garantovať, že vrcholy ohraničujúcich hrán ležia na triangulovanej ploche ale aj na ohraničujúcej obálke. Taktiež 
by mala zabrániť tvorbe úzkych trojuholníkov pri okraji triangulovanej plochy.
Nakoniec je možnosť budúcej práce na robustnejšom algoritme na uzatváranie dier.