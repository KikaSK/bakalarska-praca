\chapter*{Úvod} % chapter* je necislovana kapitola
\addcontentsline{toc}{chapter}{Úvod} % rucne pridanie do obsahu
\markboth{Úvod}{Úvod} % vyriesenie hlaviciek

Vizualizácia, teda zobrazovanie, sa už od praveku využíva ako nástroj efektívnej komunikácie.
Digitálna vizualizácia patrí medzi dôležité oblasti počítačovej grafiky, 
v minulosti využívaná najmä v medicíne, pri zobrazovaní medicínskych skenov.
Dnes má omnoho širšie využitie napríklad pri vizualizácii produktov v architektúre, 
či strojárskom priemysle, ale aj ako efektívna forma vzdelávania. 

Implicitne definované plochy sú bežnou formou reprezentácie $3D$ plôch. V počítačovej grafike 
sú veľmi populárne najmä na reprezentáciu objektov pri \textit{modelovaní}, \textit{animovaní}, 
\textit{tvorbe simulácií} a \textit{vizualizácií}. Sú aj vhodnou
voľbou pri vizualizácií medicínskych dát, ako sú skeny z počítačovej tomografie (CT) a magnetickej
rezonancie (MRI) \cite{de2015survey}.

Pri vizualizácii 3D plochy sa využívajú priame metódy, ktoré renderujú plochu 
priamo z jej matematickej reprezentácie. Pravdepodobne najznámejšou metódou priamej vizualizácie je 
metóda založená na sledovaní lúča (\textit{ray--tracing}). Táto metóda je najpresnejšia z metód vizualizácie,
je však výpočtovo náročná. Preto sa namiesto priamych metód využívajú nepriame metódy, 
ktoré spočívajú v diskrétnej aproximácii plochy, najčastejšie po častiach lineárnym 
polygónovým \textit{meshom}. V porovnaní s priamymi metódami 
vizualizácie sú nepriame metódy omnoho rýchlejšie, avšak aj menej presné. Pri trojuholníkových 
\textit{meshoch} závisí presnosť na zvolenej dĺžke hrany triangulačného trojuholníka. 

V tejto práci sa venujeme algoritmom vytvárania trojuholníkového \textit{meshu} pre 
implicitne definovanú plochu. Takýto proces nazývame triangulácia. Medzi známe triangulačné 
algoritmy patria rýchle, avšak nekvalitné metódy založené na prerozdeľovaní priestoru, 
ale aj pomalšie, kvalitnejšie metódy založené na sledovaní plochy.
My sa zameriavame na tvorbu kvalitného algoritmu založeného na sledovaní plochy
s nižším dôrazom na rýchlosť výpočtu.

V kapitole \ref{kap:plochy} predstavíme plochy zadané implicitne a priblížime niektoré ich vlastnosti.
V kapitole \ref{kap:numeric_methods} opíšeme používané numerické metódy.
V kapitole \ref{kap:triangulation} zadefinujeme trianguláciu, jej typy a niektoré, už existujúce algoritmy.
V kapitole \ref{kap:algoritmus} odprezentujeme návrh nášho algoritmu a nakoniec v 
kapitole \ref{kap:results} zhrnieme dosiahnuté výsledky. 


