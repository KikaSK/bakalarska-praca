\chapter*{Úvod} % chapter* je necislovana kapitola
\addcontentsline{toc}{chapter}{Úvod} % rucne pridanie do obsahu
\markboth{Úvod}{Úvod} % vyriesenie hlaviciek

Implicitne definované plochy sú bežnou formou reprezentácie $3D$ plôch. V počítačovej grafike 
sú veľmi populárne najmä na reprezentáciu objektov pri \textit{modelovaní}, \textit{animovaní}, 
\textit{tvorbe simulácií} a \textit{vizualizácií}. Sú taktiež veľmi vhodnou
voľbou pri vizualizácií medicínskych dát, ako sú skeny z počítačovej tomografie (CT) a magnetickej
rezonancie (MRI) \cite{de2015survey}.
Dnes majú vizualizácie implicitne definovaných plôch oveľa širšie využitie.

Vizualizácia týchto plôch je v počítačovej grafike otvorenou témou. Poznáme 
dva hlavné typy vizualizácie -- priame metódy a nepriame metódy. 

Pri priamych metódach sú plochy renderované priamo z ich 
matematickej reprezentácie. Pravdepodobne najznámejšou metódou priamej vizualizácie je 
metóda založená na sledovaní lúča (\textit{ray--tracing}). Táto metóda je najpresnejšia z metód vizualizácie,
je však veľmi výpočtovo náročná.

Pri nepriamych metódach využívame pri renderovaní diskrétnu aproximáciu zadanej plochy, 
na čo sa využívajú najmä po častiach lineárne polygónové \textit{meshe}. 
V porovnaní s priamymi metódami 
vizualizácie sú nepriame omnoho rýchlejšie, avšak aj menej presné. Pri trojuholníkových 
\textit{meshoch} závisí presnosť na zvolelnej veľkosti hrany triangulačného trojuholníka. 

V tejto práci sa venujeme algoritmom vytvárania trojuholníkového \textit{meshu} pre 
implicitne definovanú plochu. Takýto algoritmus nazývame triangulácia. Medzi známe triangulačné 
algoritmy patria rýchle avšak nekvalitné metódy založené na prerozdeľovaní priestoru, 
ale aj pomalšie avšak kvalitnejšie metódy založené na sledovaní plochy.
My sa zameriavame na tvorbu kvalitného algoritmu založeného na sledovaní plochy
s nižším dôrazom na rýchlosť výpočtu.

V kapitole \ref{kap:plochy} zadefinujeme plochy zadané implicitne a priblížime niektoré ich vlastnosti.
V kapitole \ref{kap:numeric_methods} opíšeme používané numerické metódy.
V kapitole \ref{kap:triangulation} popíšeme trianguláciu, jej typy a predchádzajúcu prácu.
V kapitole \ref{kap:algoritmus} odprezentujeme návrh nášho algoritmu a následne v kapitole 
\ref{kap:implementation} krátko zhrnieme jeho implementáciu.
Nakoniec v kapitole \ref{kap:results} odprezentujeme dosiahnuté výsledky. 


