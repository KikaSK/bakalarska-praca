\chapter{Plochy}

\label{kap:plochy} % id kapitoly pre prikaz ref

V tejto kapitole...

\section{Spôsoby zadania funkcie}

\begin{definition}
    Nech $A \subseteq \mathbb{R}^n$, $B \subseteq \mathbb{R}$ sú neprázdne množiny. 
Hovoríme, že na množine $A$ je definovaná funkcia viacerých premenných
$f$, ak je daný predpis, ktorý každej n-tici $(x_1, . . . , x_n)$
z množiny $A$ priradí práve jeden prvok $f(x_1, . . . , x_n)$ z množiny $B$.
Zapisujeme 
$$f : A \to B$$
$$(x_1, . . . , x_n) \mapsto f(x_1, . . . , x_n)$$
Ak premennú $y \in B$ vyjadríme ako funkčnú hodnotu bodov $(x_1, . . . , x_n) \in A$
$$f : y = f(x_1, . . . , x_n),$$
hovoríme o explicitnom zadaní funkcie.
\end{definition}


Pri explicitne zadanej funckií je závislá premenná $y \in \mathbb{R}$ vyjadrená ako funkčná hodnota nezávislých
premenných $(x_1, . . . , x_n) \in \mathbb{R}^n$.
\\*
Funkcia zadaná implicitne je daná pomocou funkcie $$F : \mathbb{R}^n \times \mathbb{R} \to \mathbb{R}$$
$$(x_1, . . . ,x_n, y) \mapsto F(x_1, . . . , x_n, y),$$ ktorá každému prvku z $(x_1, . . . ,x_n, y) \in \mathbb{R}^{n+1}$ 
priradí určitú hodnotu $F(x_1, . . . , x_n, y) \in \mathbb{R}$. 
Implicitná funkcia je daná nulovou hladinou tejto funkcie, teda bodmi $(x_1, . . . , x_n, y)$, 
ktoré vyhovujú implicitnej rovnici $$F(x_1, . . . , x_n, y) = 0.$$  
Táto rovnica určuje vzťah medzi závislou premennou $y \in \mathbb{R}$
a nezávislými premennými $(x_1, . . . , x_n) \in \mathbb{R}^n$. 

\begin{note}
    Ak $f : \mathbb{R}^n \to \mathbb{R}$, $f: y = f(x_1, . . . , x_n)$ je funkcia a body
    $(x_1, . . . , x_n, y) \in \mathbb{R}^n \times \mathbb{R}$ vyhovujúce implicitnej rovnici 
    $$F(x_1, . . . , x_n, y) = 0$$
    sú grafom funkcie f,
    potom je funkcia f implicitne zadaná danou rovnicou.
\end{note}

Teda pre každú explicitne zadanú funkciu existuje implicitná rovnica, ktorá ju určuje.
Naopak to však neplatí. Nie každá implicitná rovnica určuje vzťah pre jedinú explicitne zadanú
funkciu. Príkladom je implicitná rovnica pre jednotkovú kružnicu $x^2 + y^2 - 1 = 0$. 
Aj keď sa implicitné funkcie nedajú vždy vyjadriť explicitne, pre každú spojite diferencovateľnú
implicitnú funkciu platí, že na nejakom okolí takmer každého bodu, existuje explicitná funkcia
ktorou sa dá vyjadriť.
O tom nám hovorí nasledujúca veta.
\\*
\\
V našej práci sa zaoberáme algoritmami triangulácie plôch v $R^3$, teda odteraz
budeme pracovať iba s funkciami pre $n = 2$. Ďalej bude platiť, že $x$ a $y$ sú 
závislé premenné a $z$ je nezávislá premenná.


\begin{theorem}
 (o funckií danej implicitne pre $\mathbb{R}^3$)
 
 Nech funkcia $\mathbb{R}^3 \to \mathbb{R}$ je spojite diferencovateľná. 
 Nech $(x_0, y_0, z_0) \in \mathbb{R}^3$ je bod taký, že $F(x_0, y_0, z_0) = c$.
 Ak $$\frac{\partial F}{\partial z} (x_0, y_0, z_0) \neq 0$$ potom existuje okolie 
 bodu $(x_0, y_0, z_0)$ také, že ak $(x, y)$ je dostatočne blízko $(x_0, y_0)$, 
 tak v tomto okolí existuje jediná funkcia $z = z(x ,y)$ spĺňajúca implicitnú rovnicu
 $F(x, y, z(x, y)) = c$. Navyše platí $z(x_0, y_0) = z_0$.
\end{theorem}

Pomocou explicitne zadanej funkcie nevieme vyjadriť veľa reálnych objektov, 
preto je funkcia daná implicitne pri vykresľovaní povrchu veľmi obľúbená. 
Pomocou implicitne daných funkcií sa veľmi jednoducho modelujú CSG objekty 
---- plusy a minusy v članku 

\begin{note}
    Z vety o funkcií danej implicitne navyše vyplýva, že ak je $z = f(x,y)$ definovaná na nejakom 
    zúžení definičného oboru funkcie F podľa predchádzajúcej vety, tak $z$ je spojitá funkcia v 
    premennej x aj v premennej y. 
\end{note}

derivacia funkcie danej implicitne

gradient
normálový vektor

singularne body a krivky funkcie danej implicitne
zakrivenie funkcie

numericke metody
interpolacia ?
newton ?

triangulacia
delaunayova podmienka
algoritmy triangulacie

\begin{definition}
    Nech $f : \mathbb{R}^3 \to \mathbb{R}$ je.
\end{definition}





